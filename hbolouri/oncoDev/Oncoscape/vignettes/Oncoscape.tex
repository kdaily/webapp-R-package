\documentclass{article}
%% %\VignetteIndexEntry{Oncoscape Overview}
%% %\VignettePackage{Oncoscape}
\usepackage[noae]{Sweave}
\usepackage[left=0.5in,top=0.5in,right=0.5in,bottom=0.75in,nohead,nofoot]{geometry} 
\usepackage{hyperref}
\usepackage[noae]{Sweave}
\usepackage{color}
\usepackage{graphicx}
\usepackage{caption}
\usepackage{subcaption}

\definecolor{Blue}{rgb}{0,0,0.5}
\definecolor{Green}{rgb}{0,0.5,0}

\RecustomVerbatimEnvironment{Sinput}{Verbatim}{%
  xleftmargin=1em,%
  fontsize=\small,%
  fontshape=sl,%
  formatcom=\color{Blue}%
  }
\RecustomVerbatimEnvironment{Soutput}{Verbatim}{%
  xleftmargin=0em,%
  fontsize=\scriptsize,%
  formatcom=\color{Blue}%
  }
\RecustomVerbatimEnvironment{Scode}{Verbatim}{xleftmargin=2em}



\renewenvironment{Schunk}{\vspace{\topsep}}{\vspace{\topsep}}
\fvset{listparameters={\setlength{\topsep}{6pt}}}
% These determine the rules used to place floating objects like figures 
% They are only guides, but read the manual to see the effect of each.
\renewcommand{\topfraction}{.99}
\renewcommand{\bottomfraction}{.99}
\renewcommand{\textfraction}{0.0}

\title{Oncoscape} 
\author{Paul Shannon}

\begin{document} 

\maketitle

\section{Introduction and Basic Operations}

Oncoscape (though the name will soon change) is a hybrid R and Javascript web browser application, offering three features:

\begin{itemize}
    \item Interactive graphics in a web browser, for ``lay'' users to explore and analyze clinical and molecualr cancer data
    \item Analyses and data provided by R
    \item A simple well-separated software architecture (an R server, HTML/CSS/Javascript running in the browser) encouraging fast 
      and agile development.  
\end{itemize}

Taking these in turn:

\begin{itemize}
    \item Interactive graphics are built upon Javascript, jQuery, d3, and cytoscape.js. (links needed).  
    \item The data-handling and analysis capabilities are provided by R code and libraries.
    \item The software architecture separates these two rich and capable environments, connecting them
         only by passing simple JSON messages back and forth.  HTML/Javascript/CSS sprawl is tamed by
         identifying separable``components'' and by assembling html at build-time via the unix tools ``make'' and ``m4''.
\end{itemize}

The JSON messages are at present passed over websockets; alternative mechanisms could be used.  The messags five fields:

\begin{itemize}
    \item cmd:  ``fetchClinicalData'', ``calculatePCA''
    \item status: ``request'', ``response'', ``error''
    \item source:  identifies the message sender
    \item target:  indentifies the target
    \item payload:  arbitrarily complex, message-specific data
\end{itemize}

\section{Other Packages Needed}

\begin{itemize}
    \item websockets:  see \url{https://github.com/rstudio/R-Websockets}
    \item RJSONIO
    \item base64enc
    \item pls
    \item RUnit
    \item survival
    \item coin
    \item ggplot2
\end{itemize}

For all packages except websockets (which must be obtained from githbub, and built from source), 
this command will download and install the necessary packages:
\begin{Schunk}
\begin{Sinput}
> source("http://bioconductor.org/biocLite.R");
> biocLite(c("RJSONIO", "base64enc", "pls", "RUnit", "survival", "coin", "ggplot2",))
\end{Sinput}
\end{Schunk}

\section{Softare Organization and Package Structure}

The Oncoscape web app is made up of ``components'' each of which has a user interface (seen in the browser) and operations 
(sometimes performed in javascript, often carried out by sending messages to R, with results returned).  A full-featured Oncoscape assembles
a number of these components into a single web page, arranged one component per user interface tab -- though other arrangements
are possible.

The package has  inst/scripts/ subdirectories, one per component, and a few more subdirectories which combine those components
into a more-or-less integrated application (``app directories'').  This encourages independent development and testing
of the components.  Each component directory has standard files:  widget.html, code.js, index.pre, makefile, run.R.  
Issuing the 'make' command builds and runs each directory's  version of Oncoscape, either a single component
(for testing and development) or an app made of multiple components.

The traditional unix development tools ``make'' and ``m4'' assemble browser-ready HTML files, one per component or app. 
All of the HTML, CSS and Javascaript are  are built into the R package, installed along with the package, and at run time,
served up to the user's browser by a small HTTP server in the package (and provided by the websockets R package).

\section{Running Oncoscape}

In order to run the current Oncoscape webapp:

\begin{itemize}
   \item library(Oncoscape)
   \item startWebApp("tabsApp/index.hmtl", 7781L)
\end{itemize}

Socket handling needs to be improved.  Sockets are usually, but not always, released upon program termination.  An on.exit hook should help.
An adaptive strategy to find an open socket would be a boon to the user.  A utility which surveys current socket claims would be handy.

To build Oncoscape in development mode.  For example:

\begin{itemize}
  \item cd Oncoscape/inst/scripts/gbmPathways 
  \item make
\end{itemize}

The make command assembles HTML, CSS and Javascript into a single file -- 'index.html' -- the user interface.  
It builds and installs the R package, containing clinical and molecular data, and all of the operations
requested by the user interface.   Finally, make will start R, load the new package, and direct
your browser to display 'index.html'.   The Javascript and the R websockets server establish a communication 
channel over the specified port, data is loaded, the web page is rendered, and the application is ready.


\end{document}


